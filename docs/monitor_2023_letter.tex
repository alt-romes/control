%%%%%%%%%%%%%%%%%%%%%%%%%%%%%%%%%%%%%%%%%
% Plain Cover Letter
% LaTeX Template
% Version 1.0 (28/5/13)
%
% This template has been downloaded from:
% http://www.LaTeXTemplates.com
%
% Original author:
% Rensselaer Polytechnic Institute 
% http://www.rpi.edu/dept/arc/training/latex/resumes/
%
% License:
% CC BY-NC-SA 3.0 (http://creativecommons.org/licenses/by-nc-sa/3.0/)
%
%%%%%%%%%%%%%%%%%%%%%%%%%%%%%%%%%%%%%%%%%

%----------------------------------------------------------------------------------------
%	PACKAGES AND OTHER DOCUMENT CONFIGURATIONS
%----------------------------------------------------------------------------------------

\documentclass[11pt]{letter} % Default font size of the document, change to 10pt to fit more text

% \usepackage{newcent} % Default font is the New Century Schoolbook PostScript font 
%\usepackage{helvet} % Uncomment this (while commenting the above line) to use the Helvetica font

% Margins
\topmargin=-0.5in % Moves the top of the document 1 inch above the default
\textheight=9in % Total height of the text on the page before text goes on to the next page, this can be increased in a longer letter
\oddsidemargin=-10pt % Position of the left margin, can be negative or positive if you want more or less room
\textwidth=6.5in % Total width of the text, increase this if the left margin was decreased and vice-versa

% \let\raggedleft\raggedright % Pushes the date (at the top) to the left, comment this line to have the date on the right

\begin{document}

%----------------------------------------------------------------------------------------
%	ADDRESSEE SECTION
%----------------------------------------------------------------------------------------

\begin{letter}{Departamento de Informática\\
FCT NOVA\\
2829-516 Caparica\\
Portugal} 

%----------------------------------------------------------------------------------------
%	YOUR NAME & ADDRESS SECTION
%----------------------------------------------------------------------------------------

\signature{Rodrigo Mesquita} % Your name for the signature at the bottom

%----------------------------------------------------------------------------------------
%	LETTER CONTENT SECTION
%----------------------------------------------------------------------------------------

\opening{Ex.mos Senhores Professores,} 
 
Escrevo-lhes para manifestar o meu interesse e candidatar-me para apoiar o
    funcionamento de unidades curriculares, com ênfase na cadeira de Arquitetura
    de Computadores e Linguagens e Ambientes de Programação. Durante o semestre
    vou elaborar a minha dissertação sob a orientação do professor Bernardo
    Toninho e tenho este trabalho como complemento ideal ao meu percurso, dado o
    meu foco na academia (antes da indústria) e, especialmente, o meu interesse,
    entusiasmo, e gosto por ensinar.

No último semestre letivo trabalhei na cadeira de ``Introdução à Programação
    para Ciência e Engenharia'', na qual creio ter tido um bom desempenho como
    monitor de um turno prático. Adicionalmente, há dois semestres letivos,
    durante os meus estudos na Alemanha pelo programa Erasmus, consegui uma
    posição como assistente ao curso de \emph{Fundamentos da Programação
    Funcional} que expandi com material que escrevi para uma aula teórica que
    tive a oportunidade de dar como convidado. O objetivo da aula era mostrar
    padrões avançados em programação funcional no contexto de aplicações
    gráficas, nomeadamente, a variante funcional do padrão Model-View-Controller
    e a sua comparação com o paradigma, consideravelmente mais avançado,
    \emph{Functional Reactive Programming}, para motivar os alunos com exemplos
    de programas funcionais mais complexos e tentativamente impressionantes.



% No ano passado também me candidatei a esta posição, destacando o artigo que escrevi sob
%     orientação do professor Bernardo Toninho (intitulado \emph{Síntese de Programas
%     Funcionais Lineares}) e os projetos extra-curriculares com relevância para
%     FSO (de entre os quais o início de um pequeno sistema operativo e um
%     emulador do \emph{GameBoy} escritos em C).

% Acredito agora ser um candidato ainda mais forte dado experiência de lecionação
%     e projetos mais relevantes: Durante o meu presente semestre de Erasmus na
%     Alemanha obtive uma posição como assistente ao curso de \emph{Fundamentos da
%     Programação Funcional}, que estou a aumentar com material escrito por mim e
%     no âmbito do qual irei dar pelo menos uma palestra no final do curso. Ao
%     mesmo tempo, dou lições privadas de \emph{Haskell} a um estudante online,
%     escrevendo o meu próprio material teórico\footnote{e.g.
%     https://alt-romes.github.io/archive/2022-05-08-lecture-2.html} e prático.
%     Duas oportunidades talvez apenas possíveis por ser um contribuidor ativo do
%     \emph{Glasgow Haskell Compiler}
%     \footnote{https://gitlab.haskell.org/alt-romes}, um projeto \emph{open
%     source} de renome. Adicionalmente, escrevi em C um compilador para a
%     linguagem C (que gera código
%     LLVM)\footnote{https://github.com/alt-romes/c-compiler}, que mais uma vez
%     espero demostrar habilidade e entendimento nos temas relevantes à cadeira de
%     FSO.  Relembro também a minha participação em duas atividades de promoção do
%     DI: a \emph{Feira Virtual @ DI} e a palestra de introdução aos alunos do ano
%     passado na qual apresentei o meu projeto de APDC na vertente de
%     investigação.
 
Por fim, reitero o meu interesse em trabalhar no departamento, ficando
    disponível para uma conversa onde possa apresentar e prestar esclarecimentos
    sobre esta proposta.

\closing{Atentamente,}


\encl{Curriculum vitae, parecer do orientador} % List your enclosed documents here, comment this out to get rid of the "encl:"

%----------------------------------------------------------------------------------------

\end{letter}

\end{document}
