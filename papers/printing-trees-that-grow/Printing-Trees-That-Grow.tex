\documentclass{article}

\title{Printing Trees That Grow}
\author{Rodrigo Mesquita}
\date{June 2022}
% \institute{NOVA School of Science and Technology}

\begin{document}

\maketitle

\section{Introduction}

\emph{Trees That Grow}\ref{} is a programming idiom to define extensible data
types, which particularly addresses the need for decorating abstract syntax
trees with different additional information accross compiler stages. With this
newfound extensibility, we are able to share one AST data type accross compiler
stages and other AST clients --- both of which need to define their own
extensions to the datatype. This extensibility comes from using type-level
functions in defining the data types, and having the user instance them with the
needed extension.

A drawback of this extensible definition of a datatype is that few can be done
without knowing the particular instance of the datatype's extension.
This means the defined AST is, by itself, unusable.

One of the promises of extensible data types is the reduction of duplicated
code, therefore, we might be tempted to define generic functions or type-class
instances for it. In the original paper some solutions are provided

* ignore the extension points, although we no longer give the user the
    flexibility of an function or instance that takes into consideration the
    extension points they defined.
* or make use of higher order functions in the implementation, allowing for some
    custom usage of the extension points, but still restricted within the
    context of the generic implementation.

The second option, while more flexible, still isn't sufficient when faced with
the need to define a radically different implementation for a particular
constructor of the datatype, in which we might want to additionally make use of
the defined extensions.

We are then faced with the unattractive choice of either reducing duplicated
code at the cost of flexibility, or of requiring a complete implementation of the
function from any user needing that extra bit of flexibility.

This paper describes an idiom to define generic functions over the extensible
abstract syntax tree which allow drop-in definitions from the user that take
their extension instance into account.

\end{document}
