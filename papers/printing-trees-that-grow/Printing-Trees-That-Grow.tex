\documentclass{article}

\usepackage{listings}

\title{Printing Trees That Grow}
\author{Rodrigo Mesquita}
\date{June 2022}
% \institute{NOVA School of Science and Technology}

\begin{document}

\maketitle

\section{Introduction}

\emph{Trees That Grow}\ref{} is a programming idiom to define extensible data
types, which particularly addresses the need for decorating abstract syntax
trees with different additional information accross compiler stages. With this
newfound extensibility, we are able to share one AST data type accross compiler
stages and other AST clients --- both of which need to define their own
extensions to the datatype. This extensibility comes from using type-level
functions in defining the data types, and having the user instance them with the
needed extension.

As an example, here is the extensible definition of an abstract syntax tree
(AST).

\begin{lstlisting}
type Var = String
data Typ = Int
         | Fun Typ Typ

data Expr p = Lit (XLit p) Integer
            | Var (XVar p) Var
            | Ann (XAnn p) (Expr p) Typ
            | Abs (XAbs p) Var (Expr p)
            | App (XApp p) (Expr p) (Expr p)
            | XExpr !(XXExpr p) -- Constructor extension point

type family XLit p
type family XVar p
type family XAnn p
type family XAbs p
type family XApp p
type family XXExpr p
\end{lstlisting}

And an AST with no additional decorations could be extended from the above
definition as

\begin{lstlisting}
data UD

type instance XLit   UD = ()
type instance XVar   UD = ()
type instance XAnn   UD = ()
type instance XAbs   UD = ()
type instance XApp   UD = ()
type instance XXExpr UD = Void
\end{lstlisting}

A drawback of this extensible definition of a datatype is that few can be done
without knowing the particular instance of the datatype's extension.
This means the defined AST is, by itself, unusable.

One of the promises of extensible data types is the reduction of duplicated
code, therefore, we might be tempted to define generic functions or type-class
instances for it. In the original paper some solutions are provided

\begin{itemize}
    \item ignore the extension points, although we no longer give the user the
        flexibility of an function or instance that takes into consideration the
        extension points they defined.

    \item or make use of higher order functions in the implementation, allowing for some
        custom usage of the extension points, but still restricted within the
        context of the generic implementation.
\end{itemize}

The second option, while more flexible, still isn't sufficient when faced with
the need to define a radically different implementation for a particular
constructor of the datatype, in which we might want to additionally make use of
the defined extensions. We might also note that to define functions generic over
the field extension points, a lot of higher order functions or dictionaries must
be passed to the functions, and the type-class instance of an extension point is
the same regardless of the constructor its found in.

We are then faced with the unattractive choice of either reducing duplicated
code at the cost of flexibility, or of requiring a complete implementation of the
function from any user needing that extra bit of flexibility.

This paper describes an idiom to define generic functions over the extensible
abstract syntax tree which allow drop-in definitions from the user that take
their extension instance into account.

\section{Watering Trees That Grow}

We would like to construct a clever way of having generic definitions of
functions over an extensible data type, definitions which allow the extensible
data type user to override particular parts of the implementation and delegate
to the generic implementation of the function the non-overriden cases ---
allowing for a possible complete reimplementation of the instance if desired.

At first sight, a function that can default to some other implementation can
simply be a function that takes as parameter a higher-order function which is
the default implementation itself.

With a small tweak, the default implementation itself always calls the
so called \emph{override} function and pass it the actual default implementation as
an argument.

For example, if we were to write a pretty printer for the above defined AST,
which by default works regardless of the extension points, but that can be
overriden on some or all constructors, we could have

\begin{lstlisting}
override :: (Expr p -> String) -> Expr p -> String

pprDefault :: Expr p -> String
pprDefault = override $ \case 
    Lit _ i -> show i
    Var _ s -> s
    Ann _ e t -> "(" <> printE e <> ") :: (" <> printT t <> ")"
    Abs _ v e -> "λ" <> v <> "." <> printE e
    App _ f v -> "(" <> printE f <> ") (" <> printE v <> ")"
    XExpr _ -> ""
\end{lstlisting}

\end{document}
